% Created 2015-11-01 Sun 00:52
\documentclass[letterpaper, 11pt]{article}
\usepackage[utf8]{inputenc}
\usepackage[T1]{fontenc}
\usepackage{fixltx2e}
\usepackage{graphicx}
\usepackage{longtable}
\usepackage{float}
\usepackage{wrapfig}
\usepackage{rotating}
\usepackage[normalem]{ulem}
\usepackage{amsmath}
\usepackage{textcomp}
\usepackage{marvosym}
\usepackage{wasysym}
\usepackage{amssymb}
\usepackage{hyperref}
\tolerance=1000
\usepackage{palatino}
\usepackage{sidenotes}
\usepackage{algorithm}
\usepackage{algorithmic}
\usepackage[top=1in, bottom=1in, right = 0.5in, outer=3in, inner=0.5in, heightrounded, marginparwidth=2.5in, marginparsep=0.25in]{geometry}
\linespread{1.3}
\providecommand{\diff}[2]{\ensuremath{\frac{{\rm d} #1}{{\rm d} #2}}}
\providecommand{\note}[1]{\begin{margintable}{\footnotesize #1}\end{margintable}}
\providecommand{\abs}[1]{\ensuremath{\left | #1\right |}}
\providecommand{\On}[1]{\ensuremath{\mathcal{O}(h^{#1})}}
\graphicspath{{./image/}}
\author{Yisu Nie}
\date{\today}
\title{Notes on Dynamic Optimization}
\hypersetup{
  pdfkeywords={},
  pdfsubject={},
  pdfcreator={Emacs 24.4.1 (Org mode 8.2.10)}}
\begin{document}

\maketitle

\section{Initial Value Problems}
\label{sec-1}
The general form of a first order initial value problem (IVP) for ordinary differential equation (ODE) systems can be stated as follows:
\note{Different notation for differentiation:}
\begin{margintable}
\footnotesize
\begin{tabular}{ll}
 Gottfried Leibniz & $\frac{\rm dz^{{n}}}{\rm dt^{{n}}}$  \\
 Joseph Louis Lagrange & $z^{\prime}(t), z^{\prime\prime}(t),..z^{(n)}(t)$ \\
 Isaac Newton & $\dot z$, $\ddot z$, ..
\end{tabular}
\end{margintable}

\begin{subequations}
\label{eq:ivp-gen-form-def}
\begin{align}
\label{eq:ivp-gen-form-def-a}
& \diff{z}{t}= f(t,z), \qquad t \in [0,t_{f}]; \\
\label{eq:ivp-gen-form-def-b}
& z(0) = z_{0}. 
\end{align}
\end{subequations}

The dependent variable $z$ is a vector of $m$ components. The independent variable $t$ is a scalar within the specified range from 0 to $t_{f}$.If $t$ does not appear explicitly in
the governing equation $f(\cdot)$, the system is called \emph{autonomous}. Otherwise, the system is \emph{nonautomonous}. The initial state of the system is given by a known
parameter vector $z_0$. 

Numberical solution approaches deal with finite dimensional representations of Eq.\eqref{eq:ivp-gen-form-def} after discritizing the equations in the continuous interval. For that, a
mesh is introduced with a sequence of $N+1$ distant points:
\begin{equation}
\label{eq:ivp-mesh-def}
0=t_{0}<t_{1}<\cdots<t_{n-1}<t_{n}<\cdots<t_{N}=t_{f}
\end{equation} 
and the length of the $n^{th}$ step is denoted by:
\begin{equation}
\label{eq:ivp-step-size-def}
h_{n} = t_{n} - t_{n-1}, \qquad n=1,2,\ldots,N.
\end{equation} 
It's generally helpful to use Taylor's expansion to derive numerical solution procedures to Eq.\eqref{eq:ivp-gen-form-def-a}. Consider a Taylor series at $t=t_{n-1}$: 
\begin{equation}
\label{eq:ivp-taylor-series-def}
z(t_{n}) = z(t_{n-1} + h_{n}) = z(t_{n-1}) + h_{n}z'(t_{n-1}) + \frac{h^{2}_{n}}{2}z''(t_{n-1}) + \ldots + \frac{h^{p}_{n}}{k!}z^{(p)}(t_{n-1}) + \ldots,
\end{equation}
which is often trucated up to the second order:
\begin{margintable}
\footnotesize
$x=\mathcal{O}(h^{p})$ means $\exists C>0$ such that \left | x \right | \leqslant Ch^{p}.
\end{margintable}
\begin{equation}
\label{eq:ivp-taylor-series-truc}
z(t_{n}) = z(t_{n-1}) + h_{n}z'(t_{n-1}) + \frac{h^{2}_{n}}{2}z''(t_{n-1}) + \mathcal{O}(h^{2}_{n})
\end{equation}
\subsection{Basic First Order Approaches}
\label{sec-1-1}
Three well known basic approaches for IVPs are the forward Euler, backward Euler, and Trapezoidal methods.
\subsubsection{Forward Euler}
\label{sec-1-1-1}
If the expansion in Eq.\eqref{eq:ivp-taylor-series-def} is only up to the first order, we have 
\begin{margintable}
\footnotesize
We note $z(t_{n})$ as $z_{n}$ for short
\end{margintable}
\begin{equation}
\label{eq:ivp-forward-euler}
z_{n} \approx z_{n-1} + h_{n}z'_{n-1}, \qquad, n=1,2,\ldots,N.
\end{equation}
Evaluate the derivative $z'_{n-1} = f(t_{n-1},z_{n-1})$ and use the first order approaxiation, and the forward Euler formula\note{This is an explicit formula} is
\begin{equation}
\label{eq:ivp-forward-euler-step}
z_{n} = z_{n-1} + h_{n}f(t_{n-1},z_{n-1}), \qquad, n=1,2,\ldots,N.
\end{equation} 
\begin{margintable}
\footnotesize
Truncation error is obtained by inserting the analytical solution z(t) into the numerical method and dividing by the step size:
\[ T_{n} = \frac{z_{n+1} - z_{n}}{h} - f \left( t_{n}, z(t_{n}) \right)\],
and $f \left( t_{n}, z(t_{n}) \right) = z'(t_{n})$.
\end{margintable}
\subsubsection{Backward Euler}
\label{sec-1-1-2}
The backward Euler formula is also derived from Taylor series of centered at $t=t_{n}$:
\begin{equation}
\label{ivp-backward-euler}
z(t_{n}) = z(t_{n+1} - h_{n+1}) = z(t_{n+1}) - h_{n+1}z'(t_{n}) + \frac{h^{2}_{n+1}}{2}z''(t_{n}) + \mathcal{O}(h^{2}_{n+1})
\end{equation}
Similar to the previous procedure, we obtain the backward Euler formula\note{This is an implicit formula}:
\begin{equation}
\label{eq:ivp-backward-euler-step}
z_{n+1} = z_{n} + h_{n+1}f(t_{n},z_{n})
\end{equation}
\subsubsection{Trapezoidal Method}
\label{sec-1-1-3}
The Trapezoidal rule is developed by Taylor expansion centered at $t_{n-\frac{1}{2}}=t_{n-1} + \frac{h_{n}}{2}$:
\begin{subequations}
\label{eq:ivp-trapezoidal-method-taylor}
\begin{align}
\label{eq:ivp-trapezoidal-method-taylor-a}
&z(t_{n}) = z(t_{n-\frac{1}{2}} + \frac{h_{n}}{2}) = z(t_{n-\frac{1}{2}}) + \frac{h_{n}}{2}z'(t_{n-\frac{1}{2}}) + \frac{h_{n}^{2}}{2\cdot2^{2}}z''(t_{n-\frac{1}{2}}) + \mathcal{O}(h^{2}_{n}) \\
\label{eq:ivp-trapezoidal-method-taylor-b}
&z(t_{n-1}) = z(t_{n-\frac{1}{2}} - \frac{h_{n}}{2}) = z(t_{n-\frac{1}{2}}) - \frac{h_{n}}{2}z'(t_{n-\frac{1}{2}}) + \frac{h_{n}^{2}}{2\cdot2^{2}}z''(t_{n-\frac{1}{2}}) + \mathcal{O}(h^{2}_{n})
\end{align}
\end{subequations}
Substract Eq.\eqref{eq:ivp-trapezoidal-method-taylor-b} with Eq.\eqref{eq:ivp-trapezoidal-method-taylor-a} and devide by the step size $h_{n}$:
\begin{equation}
\label{eq:ivp-trapezoidal-step-a}
\frac{z_{n} - z_{n-1}}{h_{n}} = z'_{n-\frac{1}{2}} + \mathcal{O}(h^{2}_{n})
\end{equation}  
Evaluating the derivatie at the middle point and rearranging the above equation leads to
\begin{margintable}
\footnotesize
Derivative at the middle point\\
$z'_{n-\frac{1}{2}} = \frac{1}{2}\left (f(t_{n-1},z_{n-1}) + f(t_{n},z_{n})\right)$\\
\includegraphics[scale=0.7]{1-ivp-trapezoidal}
\end{margintable}

\begin{equation}
\label{eq:ivp-trapezoidal-step-b}
z_{n} = z_{n-1} + \frac{h_{n}}{2}\left (f(t_{n-1},z_{n-1}) +  f(t_{n},z_{n})\right)
\end{equation} 
\subsection{Stability and Stiffness}
\label{sec-1-2}
\subsubsection{Stability}
\label{sec-1-2-1}
Stability can be interpreted as the integration solution $z_{n=1,..,N}$ should not instigate substantial changes with small changes in the initial condition $z_{0}$. This
transaltes to the careful choice of the step size $h_{n}$ in the Euler and Trapezoidal methods.
\begin{margintable}
\footnotesize
We should be careful when using the term stability: it could refer to the stability of the prolem or the numerical method, and we exam the latter in this section. 
\end{margintable}
A linear test fucntion is often introudced for stability analysis: 
\begin{margintable}
\footnotesize
For nonlinear systems, consider the derivative (Jocobian in case of mutiple states) $\lambda = \diff{f}{z}$. 
\end{margintable}
\begin{equation}
\label{eq:ivp-stability-test-func}
z'=\lambda z,\qquad z(0) = z_{0}
\end{equation}
with the exact solution:
\begin{equation}
\label{eq:ivp-stability-test-func-sltn}
z(t) = z_{0}e^{\lambda t}
\end{equation}
Depending on the different values of the real part of $\lambda$, we have
\begin{margintable}
\footnotesize
More rigourous definition on stability is given in Chapter 2 of Ascher and Petzold~\cite{M.Ascher1998b}}
\end{margintable}
\begin{center}
\begin{tabular}{lll}
\hline
$\operatorname{Re}(\lambda) \leq 0$ & Stable & May oscillate ($\operatorname{Re}(\lambda) = 0$ and $\lambda \not = 0$)\\
$\operatorname{Re}(\lambda) < 0$ & Asymptotically Stable & Exponential decay\\
$\operatorname{Re}(\lambda) > 0$ & Unstable & Exponential growth\\
\hline
\end{tabular}
\end{center}
Take the forward Euler method as an example, we substitute the recursive formula \eqref{eq:ivp-forward-euler-step} to Eq.\eqref{eq:ivp-stability-test-func-sltn}:
\begin{equation}
\label{eq:ivp-forward-euler-stability}
z_{n} = z_{0}(1 + h\lambda)^{n}
\end{equation}
\note{Here we assume a uniform step size $h$}
The solution is stable if 
\begin{equation}
\label{eq:ivp-forward-euler-stability-condition}
\left | 1 + h\lambda \right | \leq 1 
\end{equation} 
\begin{margintable}
\footnotesize
Decay of $\abs{z_{n}} \text{requires} \abs{\frac{z_{n}}{z_{n-1}}} \leq 1$
\end{margintable}
This describes the region of absolute stability.
\begin{margintable}
\footnotesize
If a method is absolutely stable for all $\operatorname{Re}(\lambda) \leq 0$, then it is A-stable, where $h$ is no longer limtied by stability conditions.
\end{margintable}
Similar results can be derived for the backward Euler and the Trapezoidal methods:
\begin{subequations}
\begin{align}
\label{eq:ivp-stability-conditions}
&\frac{1}{\abs{ 1 - h\lambda}} \leq 1, \qquad \text{(Backward Euler)} \\
&\abs{\frac{1+h\lambda/2}{1-h\lambda/2}} \leq 1, \qquad \text{(Trapezoidal Rule)}
\end{align}
\end{subequations}

\subsubsection{Stiffness}
\label{sec-1-2-2}
The term stiffness is also used broadly to capture differential equations systems that:
\begin{itemize}
\item contain widely varying time scales, i.e., some components of the solution decay much more rapidly than others.
\item have numerical integration step size dictated by stability requirements rather than by accuracy requirements.
\item cannot be solved by explicit methods, or only extremely slowly.
\end{itemize}

\subsection{Runge-Kutta Methods}
\label{sec-1-3}
Runge-Kutta methods aim to achieve higher accuracy by using intermediate evaluation points between $t_{n}$ and $t_{n+1}$. If $s$ intermediate points are used, we call this a
s-stage Runge-Kutta formula\note{This is the explicit Runge-Kutta methed family, and it reduces to the forward Euler method when $i=1$}:
\begin{subequations}
\label{eq:ivp-rk-s-stage}
\begin{align}
&z_{n+1} = z_{n} + h\sum_{i=1}^{s}b_{i}k_{i} \\
&k_{i} = f \left( t_{n} + c_{i}h, z_{n} + h\sum_{j=1}^{i-1}a_{ij}k_{j} \right) \\
\label{eq:ivp-rk-s-stage-c}
&c_{1} = 0, \qquad c_{i} = \sum_{j=1}^{i-1}a_{ij}.
\end{align}
\end{subequations}
Here, $a_{ij}$ is the Runge-Kutta matrix, $b_{i}$ is the weights, and $c_{i}$ is the nodes. A short hand notation for the coefficients is known as the Buthcer tableau:

\begin{margintable}
\footnotesize
Derivation of 2-stage Runge-Kutta formula\\
\begin{align*}
& z_{n+1} = z_{n} + h f\left( b_{1}k_{1} + b_{2}k_{2} \right)\\
& k_{1} = f \left( t_{n}, y_{n} \right)\\
& k_{2} = f \left( t_{n} + c_{2}h, z_{n} + ha_{21}k_{1} \right)
\end{align*}
We have $c_{2}=a_{21}$ from Eq.~\eqref{eq:ivp-rk-s-stage-c}. The other coefficients are chosen to maximize the numerical accuracy by looking at the local truncation error. Consider Taylor series to $z_{n+1}$:
\begin{align*}
& z_{n+1} = z(t_{n+1}) = z \left( t_{n} + h \right)\\
& z_{n+1} = z_{n} + z'_{n}h + \frac{1}{2}z''_{n}h^{2} + \On{2}\\
& z_{n+1} = z_{n} + hf_{n} + \frac{1}{2} h^{2}\left( f_{t} + ff_{z} \right)_{n} + \On{2}    
\end{align*}
Consider Taylor series to $k_{1}$ and $k_{2}$:
\begin{align*}
& k_{1} = f_{n} \\
& k_{2} = f_{n} + h \left( c_{2}f_{t} + a_{21}ff_{z} \right)_{n} + \On{2} \\
& z_{n+1} = z_{n} + (b_{1} + b_{2})hf_{n} \\
& \qquad \quad + b_{2}h^{2}\left( c_{2}f_{t} + a_{21}ff_{z} \right)_{n} + \On{2}
\end{align*}
Compare the two expressions, we have $b_{1} + b_{2} = 1$ and $b_{2}c_{2} = b_{2}a_{21} = \frac{1}{2}$ 
\end{margintable}
\begin{center}
\begin{tabular}{c|ccccc}
$c_{1}$ &          &           &          &             & \\
$c_{2}$ & $a_{21}$ &           &          &             & \\
$c_{3}$ & $a_{31}$ & $a_{32}$  &          &             & \\
$\vdots$& $\vdots$ &           & $\ddots$ &             & \\
$c_{s}$ & $a_{s1}$ & $a_{s2}$  & $\cdots$ & $a_{s,s-1}$ & \\ \hline
        & $b_{1}$  & $b_{2}$   & $\cdots$ & $b_{s-1}$   & $b_{s}$
\end{tabular}
\end{center}
One of the most popular numerical approaches for IVPs is the classic four-stage Runge-Kutta method:
\begin{subequations}
\begin{align}
\label{eq:ivp-rk-4-stage}
& z_{n+1} = z_{n} + \frac{1}{6}h \left( k_{1} + 2k_{2} + 2k_{3} + k_{4} \right) \\
& k_{1} = f \left( t_{n}, z_{n} \right) \\
& k_{2} = f \left( t_{n} + \frac{1}{2}h, z_{n} + \frac{1}{2}hk_{1} \right) \\
& k_{3} = f \left( t_{n} + \frac{1}{2}h, z_{n} + \frac{1}{2}hk_{2} \right) \\
& k_{4} = f \left( t_{n} + h, z_{n} + hk_{3} \right) 
\end{align}
\end{subequations}


TODO implicit Runge-Kutta

\subsection{Multi-step Approaches}
\label{sec-1-4}
\section{Bundary Value Problems}
\label{sec-2}

\bibliographystyle{plain}
\bibliography{/Users/yisu/Documents/Dropbox/OptNotes/library}
% Emacs 24.4.1 (Org mode 8.2.10)
\end{document}
